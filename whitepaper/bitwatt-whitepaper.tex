\documentclass[12pt,twocolumn]{article}
\usepackage[utf8]{inputenc}
\usepackage{graphicx}
\begin{document}
\title{BitWatt - Co-ordinating Sustainable Power Generation}
\author{Hugo O'Connor, Bit Trade Labs}
\maketitle
\tableofcontents
\newpage
\section {Status of this Memo}

This document outlines the design for a decentralised collaborative organisation to promote sustainable power generation and requests discussion and suggestions for improvements. Distribution of this memo is unlimited.

\section {Acknowledgements}

The following people contributed the thinking and ideas collected in this document;

\section {Motivation}

Inefficiencies in the Australian retail power market mean residential solar energy producers are getting around 6cents per kiloWatt hour for their output, which is immediately sold back to retail consumers for around 24cents per kiloWatt hour. Individually, residential solar energy producers have limited bargaining power. The Ethereum blockchain provides a mechanism for individual producers to collectively bargain a better price for their output. If we can increase the price paid for sustainably produced energy, we can increase the incentive to produce sustainable energy and grow the green economy.

\section {Background}

Various initiatives to promote renewable energy production have been undertaken by the Australian government.  As the market has grown, technology has improved and prices of solar panels have come down. However, residential solar energy producers have not been able to get a fair price for their contributions to the grid. Power retailers, who are the benefit from

\section {Stakeholders}

Commercial and residential with solar already installed
- already in market: transitioning and augmentation

Willing to install (price has scared them off)
- newcomer: support to join

Raising awareness of collective approach
- outside/unaware: educating and raising awareness

Businesses
- strata management
- small businesses (existing and new)
- Carbon credit, renewable energy credits etc

Research and development
- sustainable technology research at Western Sydney University
- expand as new technology develops e.g. building-attached PVC - building-integrated PVC

\section {Proposed Model}

\section {Technical Considerations}


\end{document}